% Options for packages loaded elsewhere
\PassOptionsToPackage{unicode}{hyperref}
\PassOptionsToPackage{hyphens}{url}
%
\documentclass[
]{book}
\usepackage{amsmath,amssymb}
\usepackage{lmodern}
\usepackage{iftex}
\ifPDFTeX
  \usepackage[T1]{fontenc}
  \usepackage[utf8]{inputenc}
  \usepackage{textcomp} % provide euro and other symbols
\else % if luatex or xetex
  \usepackage{unicode-math}
  \defaultfontfeatures{Scale=MatchLowercase}
  \defaultfontfeatures[\rmfamily]{Ligatures=TeX,Scale=1}
\fi
% Use upquote if available, for straight quotes in verbatim environments
\IfFileExists{upquote.sty}{\usepackage{upquote}}{}
\IfFileExists{microtype.sty}{% use microtype if available
  \usepackage[]{microtype}
  \UseMicrotypeSet[protrusion]{basicmath} % disable protrusion for tt fonts
}{}
\makeatletter
\@ifundefined{KOMAClassName}{% if non-KOMA class
  \IfFileExists{parskip.sty}{%
    \usepackage{parskip}
  }{% else
    \setlength{\parindent}{0pt}
    \setlength{\parskip}{6pt plus 2pt minus 1pt}}
}{% if KOMA class
  \KOMAoptions{parskip=half}}
\makeatother
\usepackage{xcolor}
\usepackage{longtable,booktabs,array}
\usepackage{calc} % for calculating minipage widths
% Correct order of tables after \paragraph or \subparagraph
\usepackage{etoolbox}
\makeatletter
\patchcmd\longtable{\par}{\if@noskipsec\mbox{}\fi\par}{}{}
\makeatother
% Allow footnotes in longtable head/foot
\IfFileExists{footnotehyper.sty}{\usepackage{footnotehyper}}{\usepackage{footnote}}
\makesavenoteenv{longtable}
\usepackage{graphicx}
\makeatletter
\def\maxwidth{\ifdim\Gin@nat@width>\linewidth\linewidth\else\Gin@nat@width\fi}
\def\maxheight{\ifdim\Gin@nat@height>\textheight\textheight\else\Gin@nat@height\fi}
\makeatother
% Scale images if necessary, so that they will not overflow the page
% margins by default, and it is still possible to overwrite the defaults
% using explicit options in \includegraphics[width, height, ...]{}
\setkeys{Gin}{width=\maxwidth,height=\maxheight,keepaspectratio}
% Set default figure placement to htbp
\makeatletter
\def\fps@figure{htbp}
\makeatother
\setlength{\emergencystretch}{3em} % prevent overfull lines
\providecommand{\tightlist}{%
  \setlength{\itemsep}{0pt}\setlength{\parskip}{0pt}}
\setcounter{secnumdepth}{5}
\usepackage{booktabs}
\ifLuaTeX
  \usepackage{selnolig}  % disable illegal ligatures
\fi
\usepackage[]{natbib}
\bibliographystyle{plainnat}
\IfFileExists{bookmark.sty}{\usepackage{bookmark}}{\usepackage{hyperref}}
\IfFileExists{xurl.sty}{\usepackage{xurl}}{} % add URL line breaks if available
\urlstyle{same} % disable monospaced font for URLs
\hypersetup{
  pdftitle={UBCO Ecology of Campus Guide},
  pdfauthor={Jordan Katchen and Ian Sanga},
  hidelinks,
  pdfcreator={LaTeX via pandoc}}

\title{UBCO Ecology of Campus Guide}
\author{Jordan Katchen and Ian Sanga}
\date{2023-03-06}

\begin{document}
\maketitle

{
\setcounter{tocdepth}{1}
\tableofcontents
}
\hypertarget{about}{%
\chapter{About}\label{about}}

\textbf{BIOL420N Preplan -- Ecology of campus}

\emph{Jordan Katchen and Ian Sanga}

{{[}1{]} Topic}

\begin{itemize}
\tightlist
\item
  Ecology of campus

  \begin{itemize}
  \tightlist
  \item
    Checklist / Guide --- A guide that shows some cool / interesting plants, fungi, and animals that can be found on campus.

    \begin{itemize}
    \tightlist
    \item
      Ten examples from each category: plants, birds, mammals, insects, and fungi.
    \item
      Where specifically on campus to look.
    \item
      Gives Information on what you are looking for:

      \begin{itemize}
      \tightlist
      \item
        Common name
      \item
        Latin name
      \item
        Picture
      \item
        Written description
      \item
        Special information on invasive species
      \item
        Call / Footprint (if applicable)
      \item
        Indigenous significance (if applicable)
      \end{itemize}
    \item
      Highlight endangered, threatened, and invasive species that can be found in the community.
    \end{itemize}
  \end{itemize}
\end{itemize}

{{[}2{]} Audience}

People on the UBCO campus who are interested in local ecology. Mainly, undergraduate students looking to improve their knowledge of local ecology and/or get into the hobby of naturalism. While students in Biology, Zoology, EESC, and Ecology might already be aware of the species around them and are interested, it would be good to target people in different programs who don't have that knowledge integrated into their coursework or previous interest. For example, even though we are both in the life sciences there's really no required courses aside from first year biology and one of the three options in second year that touch on ecology so unless they take it as an elective many upper year Biochem/Microbio students may not remember much.

{{[}3{]} Motivation}

Entertainment, Engagement, Education

Show people the interesting wildlife around them and hopefully help them understand how rich the ecosystem is even in an environment as urban(?) as a campus; foster a love for ecology in people by encouraging them to look closer at the area they live in. Encourage community members to be more mindful about how they interact with the environment, notably the trails behind campus which might be more vulnerable to disturbance since they aren't as developed. Many interesting species around campus that the unobservant student may never think about throughout their degree (Did you know there are turtles in the pond behind EME?)

{{[}4{]} Approach}

Create website highlighting x number of interesting animals, plants, fungi, bugs, etc

``Success'' can be defined by a Google Form link from the website asking people if they have previously gone birdwatching etc; how many species they have already seen on campus, and if they are now more open to being naturalists and being on the look out for flora and fauna on campus

We will start by trying to use \href{https://docs.github.com/en/pages}{GitHub Pages} and \href{https://shiny.rstudio.com}{RShinny} to create our website. We will look for additional software where needed. We may also use the databases iNaturalist and GBIF to provide information and data on species.

We will also consult local expert knowledge from: Jason Pither, David Ensing, Robert Lalonde. These experts will help provide advice on which local organisms to highlight.

Have contacted an ecology lab working in the Okanagan for ideas and will get back to us on potentially interesting ideas and species for our website; one of the former grad students in the lab is now doing his PhD focused on citizen science and I would think that synergizes with out current goal of encouraging the campus community to care more about ecology

Social media promotion can be through Stories on Instagram or links from pages; I can make graphics on Canva and Biorender to either help with the campaign to actually get people onto the website as well as for actual content onto the website itself

{{[}5{]} Targeting approach}

We are planning to use course unions, student clubs, and maybe even FOS instagram.

CCU, BCCU, and other science-y course unions should be easy sells on promoting the website on their Instagram (If not I have the log-in for the CCU insta). Mainly targeting the campus community so advertising on campus-focused social media accounts would be the easiest way to reach out. Wildlife Society is present and active on campus and on Instagram - could reach out for help or collaboration, they already do some citizen science work around the community.

iGEM is another potential ``partner'' though they aren't as ecology-focused. We could also try targeting incoming international students and/or exchange students, as this is a great opportunity to explore campus and get acquainted with the local ecology.

\hypertarget{animals}{%
\chapter{Animals}\label{animals}}

\hypertarget{redtailed-hawk}{%
\section{Redtailed Hawk}\label{redtailed-hawk}}

\hypertarget{great-horned-owl}{%
\section{Great Horned Owl}\label{great-horned-owl}}

\hypertarget{northern-pacific-tree-frog}{%
\section{Northern Pacific Tree Frog}\label{northern-pacific-tree-frog}}

\hypertarget{painted-turtle}{%
\section{Painted Turtle}\label{painted-turtle}}

\hypertarget{mule-deer}{%
\section{Mule Deer}\label{mule-deer}}

\hypertarget{insects}{%
\chapter{Insects}\label{insects}}

\hypertarget{cyphocleonus-achates}{%
\section{Cyphocleonus Achates}\label{cyphocleonus-achates}}

\hypertarget{european-mantis}{%
\section{European Mantis}\label{european-mantis}}

\hypertarget{asian-lady-bug}{%
\section{Asian Lady Bug}\label{asian-lady-bug}}

\hypertarget{leopard-slug}{%
\section{Leopard Slug}\label{leopard-slug}}

\hypertarget{california-broad-necked-beetle}{%
\section{California Broad Necked Beetle}\label{california-broad-necked-beetle}}

\hypertarget{plantsfungi}{%
\chapter{Plants/Fungi}\label{plantsfungi}}

\hypertarget{horse-chestnut}{%
\section{Horse Chestnut}\label{horse-chestnut}}

\hypertarget{bonnets}{%
\section{Bonnets}\label{bonnets}}

\hypertarget{silvery-cinquefoil}{%
\section{Silvery Cinquefoil}\label{silvery-cinquefoil}}

\hypertarget{bittersweet-nightshade}{%
\section{Bittersweet Nightshade}\label{bittersweet-nightshade}}

\hypertarget{wolf-lichen}{%
\section{Wolf Lichen}\label{wolf-lichen}}

  \bibliography{book.bib,packages.bib}

\end{document}
